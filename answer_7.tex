\documentclass[a4paper, 11pt]{article}
\usepackage[utf8]{inputenc}
\usepackage[T1]{fontenc}
\usepackage[margin=3cm]{geometry}
\usepackage[shortlabels]{enumitem}
\usepackage{xcolor}
\usepackage{kvoptions}
\usepackage{amsmath}
\usepackage{amssymb}
\usepackage[title={Computer Graphics},
	prof={Piotr Didyk},
	set=7,
	authors={Arnaud Fauconnet \& Francesco Costa},
	season=Autumn,
	year=22]{usi-answers}

\usepackage{listings}
\lstset{language=Pascal}

\usepackage{pgfplots}
\pgfplotsset{compat=1.15}
\usepackage{mathrsfs}
\usetikzlibrary{arrows}
\pagestyle{empty}
\newcommand{\degre}{\ensuremath{^\circ}}
\newcommand{\ceil}{\text{ceil}}
\newcommand{\abs}{\text{abs}}

\begin{document}
\maketitle

\section*{Exercise 1 [7 points]}

To get the projected point $p'_1, p'_2$, we can use a transformation matrix $T$
similar to the one found in exercise 4 of assignment 4.

In our case, we want to project the point on the plane $z=1$, therefore our
matrix would be
\newcommand{\T}{\begin{pmatrix}
		1 & 0 & 0 & 0 \\
		0 & 1 & 0 & 0 \\
		0 & 0 & 1 & 0 \\
		0 & 0 & 1 & 0 \\
	\end{pmatrix}}
$$ T = \T $$

So
$$ p'_1 = T \cdot p_1 = \T \cdot \begin{pmatrix} -1\\-1\\2\\1 \end{pmatrix} =
	\begin{pmatrix}  -1\\-1\\2\\2 \end{pmatrix}  = \begin{pmatrix} -0.5 \\-0.5\\1\\1
	\end{pmatrix}  = (-0.5, -0.5, 1)$$
$$ p'_2 = T \cdot p_2 = \T \cdot \begin{pmatrix} 7\\1\\10\\1 \end{pmatrix} =
	\begin{pmatrix}   7\\1\\10\\10  \end{pmatrix}  = \begin{pmatrix} 0.7 \\0.1\\1\\1
	\end{pmatrix}  = (0.7, 0.1, 1)$$

The size of the single pixel is
$$ s = \frac{\tan^{-1}\left(\frac{\pi}{4}\right)}{5} = \frac{1}{5}$$

The indices $i_1, i_2$ which represent the index of $p'_1$ and $p'_2$
respectively, can be calculated by expressing the point relative to the top-left
corner of the image and then dividing the coordinates by $s$. We then take the
absolute value of the coordinates, since indices are positive.


% $$ i_1 = (-0.5, -0.5) * \frac{\frac{1}{5}}{10} = (3, 8), \qquad i_2 = (9, 5) $$
$$ p''_1 = (-0.5, -0.5) - (-1, 1) = (0.5, -1.5) $$
$$ i_1 = \ceil(p''_1 \cdot 5) = (3, 8) $$

$$ p''_2 = (0.7, 0.1) - (-1, 1) = (1.7, -0.9) $$
$$ i_2 = \ceil(p''_2 \cdot 5) = (9, 5) $$

We define the implicit line equation as
% \begin{align*} 
% F(x, y) & = ydx - xdy + x_1dy - y_1dx =\\
%         & = 6y - 3x + 3\cdot 3 - 8 \cdot 6 =\\
%         & = 6y - 3x - 39
% \end{align*}
\begin{align*}
	F(x, y) & = ydx - xdy + x_1dy - y_1dx =            \\
	        & = 6y - (-3)x + 3\cdot (-3) - 8 \cdot 6 = \\
	        & = 6y + 3x - 57
\end{align*}
where
% $$ dx = \abs(x_2 - x_1) = 6, \qquad dy = \abs(y_2 - y_1) = 3 $$
$$ dx = x_2 - x_1 = 6, \qquad dy = y_2 - y_1 = -3 $$

To decide which pixel to color, we look at the sign of $F(M)$, where $M$ is the
midpoint between the two possible pixels.

We repeat this process for all six pixels separating $p'_1$  and $p'_2$

$$ F(4, 7.5) = 0  $$
We thus color the pixel $(4, 7)$
$$ F(5, 6.5) = -3 < 0 $$
We thus color the pixel $(5, 7)$
$$ F(6, 6.5) = 0  $$
We thus color the pixel $(6, 6)$
$$ F(7, 5.5) = -3 < 0  $$
We thus color the pixel $(7, 6)$
$$ F(8, 5.5) = 0  $$
We thus color the pixel $(8, 5)$

Finally, we color $(9, 5)$ because it is $p'_2$.


\begin{itemize}
	\item The z-value of the horizontal midpoint $(6, 6)$ has vertical
	      barycentric coordinate $\lambda = \frac{1}{3}$ (formula taken from slide
	      9 of "11 - Perspective Interpolation")
	      \begin{align*}
		      p^z & = \frac{1}{ (1-\lambda)  \frac{1}{p_1^z} + \lambda \frac{1}{p_2^z} } \\
		          & = \frac{1}{ (1- \frac{1}{3})  \frac{1}{2} + \frac{1}{3} \frac{1}{10}
		      }                                                                          \\
		          & = \frac{1}{ \frac{6}{30} }                                           \\
		          & = 5
	      \end{align*}

	\item The color of the horizontal midpoint can be calculated with the
	      attribute formula (formula taken from slide
	      10 of "11 - Perspective Interpolation")
	      $$ A = \left((1 - \lambda) \frac{A_1}{p_1^z} + \lambda \frac{A_2}{p_2^z}\right)
		      \cdot p^z $$
	      where $A$ is the attribute (in our case the color of the pixel) of $p$.
	      Thus
	      \begin{align*}
		      A & = \left(\left( 1 - \frac{1}{3} \right) \cdot
		      \begin{pmatrix} 1\\ 0\\ 0 \end{pmatrix}
		      \cdot \frac{1}{2} + \frac{1}{3} \cdot
		      \begin{pmatrix} 0\\ 1\\0 \end{pmatrix}
		      \cdot \frac{1}{10} \right) \cdot 5 =             \\
		        & = \left(
		      \begin{pmatrix} \frac{1}{3} \\ 0 \\ 0 \end{pmatrix}
		      +
		      \begin{pmatrix} 0 \\ \frac{1}{30} \\ 0 \end{pmatrix}
		      \right) \cdot 5                                  \\
		        & =
		      \begin{pmatrix} 1/3 \\ 1/30 \\ 0 \end{pmatrix}
		      \cdot 5                                          \\
		        & =
		      \begin{pmatrix} 5/3 \\ 1/6 \\ 0 \end{pmatrix}
	      \end{align*}

\end{itemize}

\section*{Exercise 2 [8 points]}

We remind the definition of barycentric coordinates and it's relation to the
areas of the subtriangles.
We have that
$$ \lambda_i = \frac{w_i}{A} $$
where $w_i$ is the signed area of the subtriangle opposite to vertex $p_i$ and
$A$ the total area of the triangle.

We can now express the signed areas $w_i$ as a function of $p = (x, y)$
\begin{align*}
	2 \cdot w_1(x, y) & = \left\| ( p_{2} - p ) \times ( p_{3} - p ) \right\|     \\
	                  & = \det \begin{pmatrix}
		                           p_{2x} - x & p_{3x} - x \\
		                           p_{2y} - y & p_{3y} - y \\
	                           \end{pmatrix}                            \\
	                  & = (p_{2x} - x ) (p_{3y} - y) - (p_{2y} - y) (p_{3x} - x)  \\
	                  & = p_{2x}p_{3y} - y p_{2x} - xp_{3y} + xy - p_{3x}p_{2y} +
	y p_{3x} + xp_{2y} - xy                                                       \\
	                  & = x(p_{2y} - p_{3y})  + y (p_{3x} - p_{2x}) +
	p_{2x}p_{3y} - p_{3x}p_{2y}
\end{align*}
We have therefore the final formula
$$ w_1(x, y) =  \frac{1}{2} \bigl(x(p_{2y} - p_{3y})  + y (p_{3x} - p_{2x}) + p_{2x}p_{3y} - p_{3x}p_{2y}\bigr)$$

The formulas for $w_2$ and $w_3$ can easily be found if we look at the indexes
present in the formula for $w_1$ since
$$ w_i(x, y) = \left\| ( p_{i+1} - p ) \times ( p_{i-1} - p ) \right\|$$
hence
$$ w_2(x, y) =  \frac{1}{2} \bigl(x(p_{3y} - p_{1y})  + y (p_{1x} - p_{3x}) + p_{3x}p_{1y} - p_{1x}p_{3y}\bigr)$$
$$ w_3(x, y) =  \frac{1}{2} \bigl(x(p_{1y} - p_{2y})  + y (p_{2x} - p_{1x}) + p_{1x}p_{2y} - p_{2x}p_{1y}\bigr)$$

If we want to find the barycentric coordinates of point $(x + 1, y)$, we see that
$$ \lambda_i(x + 1, y) = \frac{w_i(x+1, y)}{A} $$
We notice easily that
\begin{align*}
	w_1(x+1, y) & =
	\frac{1}{2} \bigl((x+1)(p_{2y} - p_{3y})  + y (p_{3x} - p_{2x}) +
	p_{2x}p_{3y} - p_{3x}p_{2y}\bigr)                         \\
	            & = w_1(x, y) + \frac{1}{2} (p_{2y} - p_{3y})
\end{align*}
and similarly
$$ w_2(x + 1, y) =  w_2(x, y) + \frac{1}{2} (p_{3y} - p_{1y})$$
$$ w_3(x + 1, y) = w_3(x, y) + \frac{1}{2} (p_{1y} - p_{2y})$$

Finally, the barycentric coordinates of point $(x+1, y)$ are
$$ \lambda_i(x+1, y) = \lambda_i(x, y) + \frac{1}{2A}\left(p_{(i+1)y} - p_{(i-1)y}\right) $$
which is pretty good since we already computed $\lambda_i(x, y)$

The same reasoning can be applied to find the barycentric coordinates of point $(x, y+1)$,
since
\begin{align*}
	w_1(x, y+1) & =
	\frac{1}{2} \bigl(x(p_{2y} - p_{3y})  + (y+1) (p_{3x} - p_{2x}) +
	p_{2x}p_{3y} - p_{3x}p_{2y}\bigr)                         \\
	            & = w_1(x, y) + \frac{1}{2} (p_{3x} - p_{2x})
\end{align*}
and similarly
$$ w_2(x, y + 1) =  w_2(x, y) + \frac{1}{2} (p_{1x} - p_{3x})$$
$$ w_3(x, y + 1) = w_3(x, y) + \frac{1}{2} (p_{2x} - p_{1x})$$

So, the barycentric coordinates of point $(x, y+1)$ are
$$ \lambda_i(x, y+1) = \lambda_i(x, y) + \frac{1}{2A}\left(p_{(i-1)x} - p_{(i+1)x}\right) $$
which is pretty good since we already computed $\lambda_i(x, y)$

\begin{lstlisting}
function BoundingBox(p1, p2, p3)
    x_min =  min(p1.x, p2.x, p3.x)
    y_min =  min(p1.y, p2.y, p3.y)
    x_max =  max(p1.x, p2.x, p3.x)
    y_max =  max(p1.y, p2.y, p3.y)
    return Point(x_min, y_min), Point(x_max, y_max)
end


function RasterizeTriangle(p1, p2, p3)
    p_min, p_max = BoundingBox(p1, p2, p3)
    A = Area(p1, p2, p3)

    triangle_pixels = []
        

    // start with pixel (x_min - 1, y_min - 1)
    init_lambda_1, init_lambda_2, init_lambda_3 = Baryncentric(
        p_min.x - 1,
        p_min.y - 1,
        p1, p2, p3
    )

    for y = p_min.y to p_max.y
        // from (x, y) -> (x, y + 1)
        lambda_1 = init_lambda_1 + 1/(2*A)* (p3.x - p2.x)
        lambda_2 = init_lambda_2 + 1/(2*A)* (p1.x - p3.x)
        lambda_3 = init_lambda_3 + 1/(2*A)* (p2.x - p1.x)

        // save the coordinates of pixel (x_min - 1, y)
        init_lambda_1 = lambda_1
        init_lambda_2 = lambda_2
        init_lambda_3 = lambda_3

        for x = p_min.x to p_max.x
            // from (x, y) -> (x + 1, y)
            lambda_1 = lambda_1 + 1/(2*A)* (p2.y - p3.y)
            lambda_2 = lambda_2 + 1/(2*A)* (p3.y - p1.y)
            lambda_3 = lambda_3 + 1/(2*A)* (p1.y - p2.y)

            // save the pixel if it is inside the triangle
            if All_Non_Negative(lambda_1, lambda_2, lambda_3)
                triangle_pixels.push( (x, y) )
            end
        end
    end

    return triangle_pixels
end
\end{lstlisting}

\end{document}
